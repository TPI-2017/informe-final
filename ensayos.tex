\clearpage



% ==============================================================
\clearpage
\part{Ensayos y mediciones}\label{part:ensayos}

Tras haber realizado los prototipos de los circuitos sobre protoboards, antes de comenzar con la implementación del mismo en PCB se realizaron pruebas sobre los módulos para determinar su consumo y así estimar la cantidad de módulos máximos a encadenar.

Como estimación inicial se tuvo en consideración la ecuación presentada en la hoja de datos del MAX7219 \cite{MAX7219}. Dicha ecuación tiene en consideración las variables que pueden llegar a producir más incertidumbre en el valor de potencia de consumo del circuito, como la tensión de operación de los LEDs.

A través de esta ecuación entonces se pudo estimar con bastante precisión la corriente estimada del circuito de cada módulo esclavo. Dicho cálculo arrojó un valor de alrededor de 1,5 mA. Multiplicando este valor por 5V que alimentan el circuito y redondeando generosamente se puede asumir que cada esclavo tendrá un consumo de unos 8 mW.

El circuito maestro tiene un consumo relativamente fijo, que pudo obtenerse midiendo la corriente entre la fuente de alimentación y el pin de alimentación del módulo. El valor brindado era de 8 mA cuando no se tenía conectado ningún esclavo, y aproximadamente 10 mA cuando se contaba con un esclavo, valor aceptablemente aproximado al que se había alcanzado previamente.

Con el fin de reducir el riesgo para los multímetros utilizados, durante la implementación de otro prototipo de módulo esclavo se decidió conectar provisoriamente un resistor pequeño en la entrada de la alimentación. Midiendo la tensión sobre ese resistor y dividiendolo por su resistencia se sabe la corriente que circula por el circuito esclavo. Se utilizó inicialmente un resistor de 680 ohms, en el que la corriente observada difería ampliamente de la estimada y medida por otros medios, pero las luces de la matriz se veían mucho más tenues, indicio de que la magnitud de la corriente podía estar viéndose perjudicada por un valor muy alto de resistencia. Se procedió entonces a utilizar una resistencia mucho mas pequeña, de 10 ohms. Esta vez los valores coincidieron con los previamente obtenidos.

Después de hacer las placas pero antes de soldar los componentes, se posicionó cada uno en los respectivos huecos de la placa en los que iría soldados con el estaño.

Luego se procedió a medir continuidad con el multímetro entre los pines de los componentes y la capa de GND, mientras se movía suavemente los pines de los componentes. De esta forma se pudo saber antes del proceso de soldadura que en determinados casos, los componentes estaban en contacto con GND sin que así el esquemático lo indicara. La solución fue corroer el cobre hasta removerlo en un diámetro de aproximadamente un milímetro adicional al ya existente. Luego se volvía a colocar el componente y buscar continuidad con GND alterando la posición de los terminales. Este procedimiento fue realizado para todos los componentes.

En los casos en que la terminal de un componente debiera estar en contacto con GND, lo que se buscaba cumplir era justamente lo opuesto, es decir, que con solo apoyar el terminal en el agujero este sea tan estrecho que permita que la pata del componente esté en contacto con la placa de cobre de la capa de GND. En algunos casos esto no bastaba, por lo que fue necesario realizar pequeñas soldaduras del lado de la capa GND. Esto garantiza la correcta conectividad de los componentes en la placa pese al obstáculo de la capa GND.


\section{Conclusiones}

Pudo comprobarse a través de este proyecto la viabilidad tanto en términos de hardware como de software del diseño planteado como implementación del cartel programable de LEDs. 
La totalidad del protocolo de comunicación pudo ser implementada de acuerdo a lo pautado en los objetivos iniciales, y en tiempo y forma respecto al cronograma propuesto. El desarrollo del hardware se vio demorado por falta de disposición de recursos, ya que algunos componentes no se encuentran disponibles en todas las tiendas de electrónica (Node MCU y MAX7219). Aún así pudo implementarse un prototipo del circuito maestro y dos prototipos de los circuitos esclavos, con las debidas interfaces para las conexiones externas e internas, tanto de datos como de alimentación, y cumpliendo los patrones de diseño recomendados por el personal técnico de la Facultad. Debido a que estos prototipos fueron desarrollados con la finalidad de demostrar la viabilidad del sistema, no se incluyó en el proyecto la realización de un gabinete para el mismo.

En términos de consumo, se comprobó que la potencia consumida por el circuito coincidía con la calculada mediante los datos indicados en la hoja de datos del MAX7219. Dicho consumo es relativamente bajo, por lo que no se lo considera un impedimento relevante a la hora de conseguir nuevas fuentes de alimentación para el sistema. 

En cuanto a posibles cambios futuros para el cartel se podría sugerir la implementación de una API Web que permita que tanto sitios web como apps puedan obtener el mensaje mostrado en el cartel en ese momento. De esta manera el alcance del mensaje podría ser mayor comunicándose a varios medios desde un mismo lugar.

Otro buen cambio sería un jumper adicional en el módulo esclavo, que separe la alimentación de la tensión del circuito de la entrada de tensión de la interfaz. De esta manera, el consumo del circuito podría ser medido más facilmente, ayudando a detectar problemas.

Además podría cambiarse el tipo de pines elegido para la conexión de las interfaces, manteniendo los de salida como pines macho y reemplazando los de entrada con pines hembra, permitiendo así distinguir visualmente con más facilidad en qué sentido están conectados los módulos.
Aunque no estaba planificado en el proyecto inicial el equipo también adaptó un cargador móvil en desuso para alimentar el circuito.

El presupuesto del proyecto cumplió con exactitud con lo planteado en la sección homónima del anexo.
% Explicar el grado de cumplimiento de objetivos planteados para el trabajo.
% Evaluar y destacar el cumplimiento y desvíos del cronograma de tareas presentados en el informe inicial
% Describir claramente la actividad de cada integrante del grupo, evaluar las horas invertidas por cada uno y calcular las horas de ingeniería total
% Analizar el presupuesto que se ha invertido y el presupuesto final del proyecto incluyendo las horas de ingeniería consumidas.
