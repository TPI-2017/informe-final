\clearpage



% ==============================================================
\clearpage
\part{Ensayos y mediciones}\label{part:ensayos}
\section{Conclusiones}

Pudo comprobarse a través de este proyecto la viabilidad tanto en términos de hardware como de software del diseño planteado como implementación del cartel programable de LEDs. 
La totalidad del protocolo de comunicación pudo ser implementada de acuerdo a lo pautado en los objetivos iniciales, y en tiempo y forma respecto al cronograma propuesto. El desarrollo del hardware se vio demorado por falta de disposición de recursos, ya que algunos componentes no se encuentran disponibles en todas las tiendas de electrónica (Node MCU y MAX7219). Aún así pudo implementarse un prototipo del circuito maestro y dos prototipos de los circuitos esclavos, con las debidas interfaces para las conexiones externas e internas, tanto de datos como de alimentación, y cumpliendo los patrones de diseño recomendados por el personal técnico de la Facultad.

En términos de consumo, se comprobó que la potencia consumida por el circuito coincidía con la calculada mediante los datos indicados en la hoja de datos del MAX7219. Dicho consumo es relativamente bajo, por lo que no se lo considera un impedimento relevante a la hora de conseguir nuevas fuentes de alimentación para el sistema. 

Aunque no estaba planificado en el proyecto inicial el equipo también adaptó un cargador móvil en desuso para alimentar el circuito.

El presupuesto del proyecto cumplió con exactitud con lo planteado en la sección homónima del anexo.
% Explicar el grado de cumplimiento de objetivos planteados para el trabajo.
% Evaluar y destacar el cumplimiento y disvíos del cronograma de tareas presentados en el informe inicial
% Describir claramente la actividad de cada integrante del grupo, evaluar las horas invertidas por cada uno y calcular las horas de ingeniería total
% Analizar el presupuesto que se ha invertido y el presupuesto final del proyecto incluyendo las horas de ingeniería consumidas.
