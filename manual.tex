\section{Manual de usuario} \label{sec:manual-usuario}

\subsection{Primer uso}
Para el primer uso, es necesario tener disponible una red WiFi con los siguientes parámetros. Esto se puede conseguir utilizando un celular o una notebook, configurandola como punto de acceso, como hacer esto no se cubrirá en esta guia.

\begin{description}
	\item[Nombre de red (SSID): ] TP12017G7
	\item[Contraseña de red: abcd1234]
\end{description}

El microcontrolador cuando se prende por primera vez, tendrá un mensaje predeterminado y se conectará a la siguiente red predefinida:

La contraseña de acceso por defecto será \enquote{1234}. Ésta deberá ser modificada rápidamente en el primer uso, para evitar que un atacante lo haga antes. Este procedimiento se explica en la sección \ref{sec:guia-password}.

Como siguiente paso, se debe cambiar la red a la que se conecta el cartel a una red WiFi fija, a la que siempre se conectará el cartel al encenderse. Este procedimiento se explica en la sección \ref{sec:guia-wifi}.

%TODO completar las siguientes personas

\subsection{Conexión al cartel}\label{sec:guia-conexion}

\subsection{Cambio de contraseña de acceso}\label{sec:guia-password}

\subsection{Cambio de mensaje al cartel}\label{sec:guia-texto}

\subsection{Cambio de red WiFi}\label{sec:guia-wifi}

\subsection{Reestablecimiento de configuración}\label{sec:guia-reset}