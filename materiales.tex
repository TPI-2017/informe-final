\section{Lista de materiales}

	El cartel cuenta con un único módulo funcional maestro, y \texttt{N} módulos esclavos. Los precios están en pesos al momento de la ejecución de su respectiva compra. En el presupuesto no se tuvieron en cuenta los componentes e insumos varios (cables, gabinete, estaño, etc), ni el costo de impresión en el papel fotográfico para el PCB.

\begin{table}[ht]
	\centering
	\caption{Presupuesto módulo Master}
	\begin{spreadtab}{{tabular}{cccc}}
		@ Item							& @ Unidades& @ Precio unidad (\$)	& @ Subtotal (\$)\\ \hline
		@ NodeMCU Esp8266				& 1			& :={200}				& b2*c2		\\
		@ PCB							& 1			& :={100}				& b3*c3		\\
		@ Transistor NPN\footnotemark	& 3	& :={4}	& b4*c4		\\
		@ Resistencia 56KOhm			& 3			& :={1}					& b5*c5		\\
		@ Resistencia 680Ohm			& 3			& :={1}					& b6*c6		\\
		@ Jumpers						& 4			& :={1}					& b7*c7		\\
		@ Jack con bornera				& 1			& :={20}				& b8*c8		\\
		@ Pulsador						& 1			& :={20}				& b9*c9		\\
		@ Tecla Rocker Switch 			& 1			& :={30}				& b10*c10	\\
		@ Regleta de 1x15 pines hembra	& 2			& :={7}					& b11*c11	\\
		@ Hoja A4 fotográfico			& 1			& :={20}				& b12*c12	\\\hline
		@ Total							& 			&						& \$ :={sum(d2:d12)}\\ \hline
	\end{spreadtab}
\end{table}

\footnotetext{Del tipo 2n2222 o similar, para el testeo se utilizó el transistor 2n2369.}

\begin{table}[ht]
	\centering
	\caption{Presupuesto módulo Esclavo}
	\begin{spreadtab}{{tabular}{cccc}}
		@ Item									& @ Unidades& @ Precio unidad (\$)	& @ Subtotal (\$)\\ \hline
		@ LED									& 64		& :={1}					& b2*c2		\\
		@ PCB									& 1			& :={100}				& b3*c3		\\
		@ MAX7219								& 1			& :={45}				& b4*c4		\\
		@ Capacitor Polarizado	10µF			& 1			& :={3}					& b5*c5		\\
		@ Capacitor no polarizado 0.1µF			& 1			& :={3}					& b6*c6		\\
		@ Resistencia 24KOhm					& 1			& :={1}					& b7*c7		\\
		@ Conector hembra 8 Pin	(25.4mm)		& 2			& :={15}				& b8*c8		\\
		@ Conector hembra 5 Pin	(25.4mm)		& 2			& :={10}				& b9*c9		\\
		@ Regleta de 1x12 pines hembra (1.27mm)	& 2			& :={7}					& b10*c10	\\
		@ Hoja A4 fotográfico					& 1			& :={20}				& b11*c11	\\\hline
		@ Total									& 			&						& \$ :={sum(d2:d11)}\\ \hline
	\end{spreadtab}
\end{table}
