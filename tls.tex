\section{Sobre Transport Security Layer}
En este apartado se discutirá brevemente sobre el protocolo de red TLS, que es utilizado en este proyecto para lograr una conexión cifrada y autenticada. Esta sección tiene como objetivo explicar al lector la necesidad y la importancia de la generación de certificados TLS, proceso que se realiza en la guía de puesta en marcha.

El protocolo TLS está diseñado para permitir a aplicaciones con arquitectura de cliente-servidor comunicarse de manera que se evita las escuchas por terceros, alteración o falsificación de datos\cite{TLS}. Esto resulta crítico para este proyecto ya que el usuario de cartel debe ingresar una contraseña de forma remota; si la conexión no fuera resistente a escuchas, entonces un atacante podría capturar la contraseña sin conocimiento del usuario legítimo y conseguir acceso al cartel.

Los objetivos del protocolo TLS son los siguientes:
\begin{description}
    \item [Seguridad criptográfica:] TLS se utiliza para establecer una conexión segura entre dos partes.
    \item [Interoperabilidad:] Dos o más programadores independientes deberían poder desarrollar aplicaciones utilizando TLS que puedan intercambiar parámetros criptográficos sin tener conocimiento del código de la otra aplicacion.
    \item [Extensibilidad:] TLS busca proveer un marco al cual se le pueden incorporar nuevos algoritmos de cifrado. Esto tiene como resultado el hecho de que al agregar un nuevo algoritmo no es necesario crear un nuevo protocolo y, por lo tanto, una nueva implementación.
    \item [Eficiencia relativa:] Las operaciones criptográficas tienden a ser intensivas en cómputo, especialmente las operaciones que trabajan sobre las claves públicas. Para esto el procolo TLS especifica un mecanismo de cacheo de sesiones para reducir el número de conexiones que se debe establecer desde cero. Por otro lado, se tomaron medidas para reducir el uso de ancho de banda en la red.
\end{description}

\subsection{Criptografía asimétrica}
La criptografía asimétrica o criptografía de clave pública es cualquier sistema criptográfico que utiliza pares de claves: claves públicas que se distribuyen ampliamente y claves privadas que son conocidas únicamente por su dueño. 

Esto logra la autenticación, es decir, que la clave pública verifique que el dato haya sido cifrado utilizando la clave privada. Haciendo esto también se obtiene el cifrado del mensaje, que implica que sólo la clave privada es capaz de decifrar el dato cifrado con la clave pública \cite{crypto}.

En este sistema criptográfico, cualquiera puede cifrar un dato arbitrario (texto plano a partir de ahora)\footnote{Se utiliza en este texto el término texto plano para cualquier dato, no necesariamente debe ser texto bajo alguna codificación como ASCII, UTF-8, etc.} con la clave pública (ya que esta disponible a cualquiera) y ese dato cifrado (texto cifrado a partir de ahora) sólo se puede obtener utilizando la clave privada para decifrarlo.

En general, los algoritmos de cifrado utilizados en sistemas de clave asimétrica son computacionalmente intensos, por lo cual TLS los utiliza únicamente para el establecimiento seguro de una clave en común a las partes. Esta clave común luego se utiliza para cifrar el resto de los datos que circulan por la conexión utilizando cifrado simétrico (un mensaje se cifra con una clave y se decifra con la misma), que es menos computacionalmente intenso.

El proceso que establece la clave en común se denomina \emph{handshake}. Este procedimiento se realiza al comienzo de una conexión TLS.

En el contexto del sistema de este proyecto, el cartel es el servidor y es quien debe tener la clave pública y la clave privada mientras que el cliente no necesita tener un elemento del sistema criptográfico.

Sin embargo, esta situación tal como está planteada no resuelve el problema de la identificación del cartel, es decir, el cliente puede conectarse al cartel con la idea de que se trata del cartel al que se quiere conectar, y el cartel puede ser en realidad un atacante haciéndose pasar por el cartel, con la intención de capturar la contraseña. Este problema se resuelve simplemente incrustando a priori a la aplicación de PC la clave pública del cartel legítimo. De esta forma, el cliente (la aplicación de PC) aborta el \emph{handshake} si las firmas públicas no coinciden. El atacante posiblemente tenga en su poder la clave pública del cartel legítimo pero no su clave privada correspondiente, sin la cual no se puede realizar el \emph{handshake}, porque en algún punto del transcurso de éste, el cliente cifrará un mensaje con la clave pública y el atacante no tendrá la clave privada para decifrarlo.