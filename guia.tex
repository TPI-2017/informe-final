
\setcounter{figure}{0}
\section{Guía de puesta en marcha}
En esta anexo se explicará paso a paso como poner en marcha el sistema. Se asume que el lector de esta guía tiene conocimientos básicos de manejo de bash en GNU/Linux.
Para proceder se necesitan los siguientes elementos:
\begin{itemize}
	\item PC corriendo Debian 9 \enquote{Stretch} con ambiente de escritorio GNOME. No es necesario que sea precisamente esta distribución de GNU/Linux pero esta guía explicara los pasos para esta distribución en particular.
	\item Un módulo maestro.
	\item Al menos un modulo esclavo.
	\item Cable Micro-USB.
	% Herramientas?
\end{itemize}

\subsection{Compilación y carga del firmware}
En esta sección se explicará como instalar el kit de desarrollo de software de Espressif y cómo cargar el firmware al SoC (System on Chip). Para esto es necesario instalar software para traer el repositorio git y compilar el compilador que se utilizará. El compilador es el gcc pero configurado para generar código para el procesador Tonsillica Xtensa LX106, que es el que utiliza el ESP8266EX.

Los comandos que comiencen con \enquote{\#} deben hacerse siendo el superusuario. Una manera de abrir una sesión 
\begin{enumerate}
	\item Descargar git para poder clonar el repositorio.
\end{enumerate}

\subsection{Compilación de la aplicación de escritorio}
\subsection{Configuración del hardware}